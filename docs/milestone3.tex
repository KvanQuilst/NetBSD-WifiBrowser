\documentclass[11pt]{article}
\setlength{\parindent}{0em}
\usepackage[margin=1in]{geometry}

\begin{document}

\begin{center}
  \textbf{\Large Milestone \#3}\\\large NetBSD Wifi Browser Project 2022\\
  Dylan Roy, Stephen Loudiana, Kevin McGrane
\end{center}


\section{General Progress Report}
Progress on the API for WifiBrowser was still slower than anticipated, but still yielded a lot of success
during the last two weeks. The delays experienced for the goals of the milestone that were not met
(passkey hashing and network configuration deletion), there was successes in other areas that counteract
the failure. As a result of changing strategies from leaning heavily on \texttt{wpa\_ctrl} for interfacing
with \texttt{wpa\_supplicant} to using File I/O and directly manipulating the configuration file has
simplified many things and fixed many of the issues experienced with the prior strategy. Moving forward,
progress should be more consistent and \textit{ideally} more substantial as the work required for the goals
missed and the upcoming goals for milestone 4 share many similarities in implementation and strategy.

Progress on CLI component has been steady with little to no blocks during development. CLI can interact 
with API component. 
The implementation of the TUI is well on its way now, the last couple weeks has been 
spent on doing some final research on \texttt{ncurses} and possible roadblocks or issues that may arise in the coming
weeks of implementation. The rest of the time has been used to create a test program to get a working example
of how to use \texttt{ncurses} and what we need to know to properly implement the needed requirements to get a 
working TUI.\\

\textbf{Accomplishments}
\begin{itemize}
  \item API can basically connect to any kind of network (some network configurations might require more
    fields than what's currently supported, but the change to support it is extremely minimal)
  \item API actually saves the configuration file
  \item API testing has been expanded and refactored for easier expansion in the future
  \item API and CLI components can interact and streamline connection processes for NetBSD
  \item TUI Scaffolding and Design
\end{itemize}

\textbf{Targets Not Met:}
\begin{itemize}
  \item API does not support passkey/password hashing yet $\rightarrow$ solution isn't as trivial as previously
    thought
  \item Targets for CLI component has been met for this milestone. Cleaner organization of CLI will need to be implemented 
    before deployment. 
  \item TUI devised testing strategy was not completed due to the majority of the efforts being put towards the
    implementation and testing of \texttt{ncurses} code in general 
\end{itemize}


\section{Git Branch Report}
All work has been performed on \texttt{master} against best practice. It might actually behoove us to
work on separate branches due to the experienced difficulties with people working on master at the same
time.


\section{Individual Progress}

\textbf{Kevin:} 5 hours
\begin{itemize}
  \item Reformatted and cleaned up time log
  \item worked more on implementation of \texttt{ncurses} in test program
  \item researched more possible roadblock when implementing the TUI with the API
  \item Completed a working \texttt{ncurses} test program printing to a screen and tasking input from users
\end{itemize}

\textbf{Dylan:} 15 hours, 55 minutes
\begin{itemize}
  \item API now has to run with \texttt{sudo} due to \texttt{wpa\_supplicant} not behaving as expected
  \item API Fixed PSK issue $\rightarrow$ \texttt{wpa\_supplicant} doesn't throw a fit anymore
  \item API Added a \texttt{cleanConfig()} function for easier testing results checking
  \item API Refactored tests to be more abstracted; will be refactored more after API is mostly complete
  \item API Started individual deletion and editing of networks; have to determine best strategy for string scanning.
  \item API Started modifying list functions to output the ssid's isolated instead of all of the extra \texttt{wpa\_supplicant}
    information $\rightarrow$ this is another case of solving the C string scanning problem
  \item API Configurations actually save now
  \item API Supports auto connect to WPA-PSK networks, and manual connection to most networks $\rightarrow$ the only thing
    preventing wider support is adding more fields to our configuration struct
  \item Initialized Milestone 3
\end{itemize}

\textbf{Stephen:} 9 hours, 17 minutes
\begin{itemize}
  \item Makefile to combine CLI and API components developed 
  \item Additional CLI file to process commands has been added
  \item API methods are now implemented in CLI component
  \item CLI has the ability to connect to \texttt{wpa\_supplicant}
  \item CLI has the ability to establish manual configuration of a network
  \item CLI has the ability to establish auto configuration of a network 
  \item CLI has the ability to list available networks 
  \item CLI has the ability to list configured networks 
\end{itemize}

\end{document}




