\documentclass[11pt]{article}
\setlength{\parindent}{0em}
\usepackage[margin=1in]{geometry}

\begin{document}

\begin{center}
  \textbf{\Large Milestone \#3}\\\large NetBSD Wifi Browser Project 2022\\
  Dylan Roy, Stephen Loudiana, Kevin McGrane
\end{center}


\section{General Progress Report}
Progress on the API for WifiBrowser was still slower than anticipated, but still yielded a lot of success
during the last two weeks. The delays experienced for the goals of the milestone that were not met
(passkey hashing and network configuration deletion), there was successes in other areas that counteract
the failure. As a result of changing strategies from leaning heavily on \texttt{wpa\_ctrl} for interfacing
with \texttt{wpa\_supplicant} to using File I/O and directly manipulating the configuration file has
simplified many things and fixed many of the issues experienced with the prior strategy. Moving forward,
progress should be more consistent and \textit{ideally} more substantial as the work required for the goals
missed and the upcoming goals for milestone 4 share many similarities in implementation and strategy.

\textbf{Accomplishments}
\begin{itemize}
  \item API can basically connect to any kind of network (some network configurations might require more
    fields than what's currently supported, but the change to support it is extremely minimal)
  \item API actually saves the configuration file
  \item API testing has been expanded and refactored for easier expansion in the future
\end{itemize}

\textbf{Targets Not Met:}
\begin{itemize}
  \item API does not support passkey/password hashing yet $rightarrow$ solution isn't as trivial as previously
    thought
  \item API does not support individual deletion of networks due to a larger delay on isolation of specific information
    with C string scanning. However, there is support for deleting all of the network configurations
\end{itemize}


\section{Git Branch Report}
All work has been performed on \texttt{master} against best practice. It might actually behoove us to
work on separate branches due to the experienced difficulties with people working on master at the same
time.


\section{Individual Progress}

\textbf{Kevin:}
\begin{itemize}
  \item
\end{itemize}

\textbf{Dylan:}
\begin{itemize}
  \item API now has to run with \texttt{sudo} due to \texttt{wpa\_supplicant} not behaving as expected
  \item API Fixed PSK issue $\rightarrow$ \texttt{wpa\_supplicant} doesn't throw a fit anymore
  \item API Added a \texttt{cleanConfig()} function for easier testing results checking
  \item API Refactored tests to be more abstracted; will be refactored more after API is mostly complete
  \item API Started individual deletion and editing of networks; have to determine best strategy for string scanning.
  \item API Started modifying list functions to output the ssid's isolated instead of all of the extra \texttt{wpa\_supplicant}
    information $\rightarrow$ this is another case of solving the C string scanning problem
  \item API Configurations actually save now
  \item API Supports auto connect to WPA-PSK networks, and manual connection to most networks $\rightarrow$ the only thing
    preventing wider support is adding more fields to our configuration struct
  \item Initialized Milestone 3
\end{itemize}

\textbf{Stephen:}
\begin{itemize}
  \item
\end{itemize}

\end{document}
