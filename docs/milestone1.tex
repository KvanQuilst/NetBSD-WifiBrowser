\documentclass[11pt]{article}
\setlength{\parindent}{0em}
\usepackage[margin=1in]{geometry}

\begin{document}

\begin{center}
  \textbf{\Large Milestone \#1}\\\large NetBSD Wifi Browser Project 2021\\
  Dylan Roy, Stephen Loudiana, Kevin McGrane
\end{center}

\section{Current State of Project}
Coming to the end of Fall Quarter 2021 - the first quarter of the project - we have reached a
state of having the basic design and requirements compiled for our project, providing a
significant portion of the foundation for beginning the low-level design and implementation phase of
development for the NetBSD Wifi Browser. 

\section{Specific Goal Status}
\textbf{Goals Met:}
\begin{itemize}
  \item Vision and Scope Document
  \item Software Requirements Specification
\end{itemize}

\textbf{Goals Not Met:}
\begin{itemize}
  \item\textbf{Group Cohesiveness} An area that we failed to excel in is our cohesiveness as
    a group. This failure is constituted of poor communication between members, an uneven
    spread of work and effort poured into the project, and an inconsistency of meeting as a
    group.
  \item\textbf{Prepared to move into development} We are not collectively at a place in our
    understanding of \texttt{wpa\_supplicant} and how to interact with it to begin the actual
    writing of code.
\end{itemize}

\section{Next Steps}
Moving into the next quarter, we need to start off with a deeper exploration of \texttt{wpa\_supplicant}
and its source code and the ways to interface with it. Shortly after, we can start building out the 
scaffolding for our API and making low level design decisions.

With this, the intention is for the group dynamic to improve, increasing the level and quality
of communication as well as spreading the work ahead of us more evenly.

\end{document}
