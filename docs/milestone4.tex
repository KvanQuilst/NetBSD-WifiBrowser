\documentclass[11pt]{article}
\setlength{\parindent}{0em}
\usepackage[margin=1in]{geometry}

\begin{document}

\begin{center}
  \textbf{\Large Milestone \#4}\\\large NetBSD Wifi Browser Project 2022\\
  Dylan Roy, Stephen Loudiana, Kevin McGrane
\end{center}


\section{General Progress Report}
Progress on the API front again has been slow, but with the minor progress
and the implementation of the passphrase hashing, it has become apparent
how close the API is to v1.0 completion. Moving into the next milestone, 
the implementation of manual configuration will be changed to both simplify
the implementation and to make configuration more expandable (and able to
better accommodate the large amount of available configuration fields). An
unfortunate piece of information discovered about \texttt{wpa\_supplicant}
was that password hashing for WPA-EAP networks is not worth while considering
that the accepted hashing algorithm does not work for passwords longer than
14 characters, requiring them to be stored in plain text as a result. This
can be somewhat circumvented with better file permissions, which will be considered
during the determination of circumventing the necessity of root privileges. The 
actual implementation of the NetBSD TUI application is now under way, the next 
few days will be spent starting the TUI and building out the design and layout while
also working more with ncurses to design an optimal layout for the TUI. Once the 
layout and design is at a place where we feel comfortable with it we will start 
to implement the connections with the API which will be the final part of the 
TUI implementation\\

\textbf{Accomplishments}
\begin{itemize}
  \item API supports passphrase hashing
  \item API has generalized and improved makefile
  \item API can edit already configured networks
  \item TUI Scaffolding and Design
  \item TUI Devise testing strategy
\end{itemize}

\textbf{Targets Not Met:}
\begin{itemize}
  \item Use API without root privileges
  \item TUI Support for current API features
  \item TUI Create Tests
\end{itemize}


\section{Git Branch Report}
One additional branch was used to implement the passphrase hashing (branch \texttt{crypto})
in order to prevent compile issues for the others while figuring out the
details of needing extra dependencies and the solution not being trivial.
Beyond this, most work was performed on \texttt{master}.


\section{Individual Progress}

\textbf{Kevin: 4.5 hours}
\begin{itemize}
  \item Completed Ncurses test file
  \item Started implementation of NetBSD Wifi Browser TUI
  \item Tested design and layout ideas
  \item Created testing scheme for TUI
\end{itemize}

\textbf{Dylan:}
\begin{itemize}
  \item Implemented the ability to edit already configured networks
  \item Determined and changed strategy for passphrase hashing
  \item Implemented passphrase hashing in API
  \item Improved Makefile
  \item Cleaned up and abstracted test file
  \item Solve lab machine issues (there were many the first week of milestone 4)
\end{itemize}

\textbf{Stephen:}
\begin{itemize}
  \item Organized CLI to be more neat
  \item Additional security testing of functions
\end{itemize}

\end{document}




