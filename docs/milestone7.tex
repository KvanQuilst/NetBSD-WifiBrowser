\documentclass[11pt]{article}
\setlength{\parindent}{0em}
\usepackage[margin=1in]{geometry}

\begin{document}

\begin{center}
  \textbf{\Large Milestone \#7}\\\large NetBSD Wifi Browser Project 2022\\
  Dylan Roy, Stephen Loudiana, Kevin McGrane
\end{center}

\section{General Progress Report}
Progress within Surf was fairly minimal this milestone. Being that it
is functinally complete, little tasks are being performed to improve
it in small ways. Along with this, the Makefiles were converted to BSDMake. 
Progress for CLI mostly involved thoroughly taking the time to understand 
the skeleton code and implement the remaining CLI methods. Some research on 
getopt_long and how I might implement it. Cannot test CLI interface on my 
computer through WWU Linux shell. Currently looking for a solution to this. 

\textbf{Accomplishments}
\begin{itemize}
  \item Surf listing output has been improved 
  \item Makefiles converted to BSDMake
  \item All API methods implemented in CLI 
\end{itemize}

\textbf{Targets Not Met:}
\begin{itemize}
  \item Cannot test CLI interface on Stephen's computer 
\end{itemize}


\section{Git Branch Report}
A branch, \texttt{TUI\_test} was made to begin implementation of the TUI.

\newpage
\section{Individual Progress}

\textbf{Kevin:}
\begin{itemize}
  \item
\end{itemize}

\textbf{Dylan:} 4 hours, 5 minutes
\begin{itemize}
  \item Improve the output for Surf's listing functions
  \item Moved the testing files for Surf into the dedicated test directory
  \item Added a testing script to automate Surf testing
  \item Converted all of Surf's Makefiles into BSDMake
  \item Modified Surf's error output to only print when \texttt{DEBUG} flag set
    at compile time $\rightarrow$ might send to a log alternatively
\end{itemize}

\textbf{Stephen:} 4 hours, 0 minutes 
\begin{itemize}
  \item All API methods for the CLI have been implemented 
  \item Thoroughly took the time to go through the code, did some research on getopt_long
	and how to implement it
  \item Cannot test CLI interface on my home console, will need to find solution. I'm not 
	sure why but I cannot connect to wpa_supplicant through WWU Linux shell. Currently 
	looking for a solution to this. 
\end{itemize}

\end{document}
