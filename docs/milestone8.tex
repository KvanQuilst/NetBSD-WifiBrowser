\documentclass[11pt]{article}
\setlength{\parindent}{0em}
\usepackage[margin=1in]{geometry}

\begin{document}

\begin{center}
  \textbf{\Large Milestone \#7}\\\large NetBSD Wifi Browser Project 2022\\
  Dylan Roy, Stephen Loudiana, Kevin McGrane
\end{center}

\section{General Progress Report}
Surf is finished, so work on an XLib GUI of the wifi browser has begun.
At the very moment, the GUI has some buttons and is almost successfully
listing available networks from a scan. CLI methods are correctly implemented 
with the help listing for each command, as well as the syntax to be used for 
getopt_long. Most of the work this week was researching how to use getopt_long 
for our program. 

\textbf{Accomplishments}
\begin{itemize}
  \item Started a GUI: Hangten
  \item Help methods for CLI, program loop fixed, exit method, reorganizing of cli
\end{itemize}

\textbf{Targets Not Met:}
\begin{itemize}
  \item 
\end{itemize}


\section{Git Branch Report}

\newpage
\section{Individual Progress}

\textbf{Kevin:}
\begin{itemize}
  \item 
\end{itemize}

\textbf{Dylan:}
\begin{itemize}
  \item More Surf error stuff
  \item Started XLib GUI
  \item Buttons!
  \item Almost listing scanned networks
\end{itemize}

\textbf{Stephen: 4 hours, 13 minutes}
\begin{itemize}
  \item Implemented CLI methods to use getopt_long syntax
  \item Did some research on getopt_long and how to implement it
  \item Fixed help listings for each command and fixed command loop 
  \item Most of the work this week involved developing a plan to use 
	getopt_long in our program. To which I am fully confident in 
	implementing. 
\end{itemize}

\end{document}
