\documentclass[11pt]{article}
\setlength{\parindent}{0em}
\usepackage[margin=1in]{geometry}

\begin{document}

\begin{center}
  \textbf{\Large Final Report}\\\large NetBSD Wifi Browser Project 2022\\
  Dylan Roy, Stephen Loudiana, Kevin McGrane
\end{center}

\section*{Summary of Project}
The NetBSD Wifi Browser Project was executed in order to bring a way
to simplify the wifi connection process in NetBSD-base.
To support this goal, a library was developed called \texttt{Surf} to
handle the interfacing with \texttt{wpa\_supplicant} and handle the
configuration of networks to the extent necessary for a complete connection.
The \texttt{Surf} library is intended to be used in the development of
user interfaces that handle interactions with wifi connections, including
connecting, disconnecting, forgetting, and manually configuring how
\texttt{wpa\_supplicant} is to connect to a network.
As an example of how \texttt{Surf} could be used in a user interface, as
well as to provide interfaces for NetBSD-base to connect to wifi access points, 
a CLI (command line interface), a TUI (terminal user interface), and a GUI
(graphical user interface) were in development in addition to the \texttt{Surf} library. 
A fully interactive command line interface is included in the Wifi Browser Project. The 
CLI can interact with all methods of the Surf library. These methods help users interact 
with the configuration file. The CLI automatically connects users to \texttt{wpa\_supplicant} 
at startup and allows users to add, edit, or delete networks from the configuration file. 
In addition to listing available and configured networks, connecting and disconnecting from 
already existing connections, auto configuration of networks and eap networks, clearing 
the screen and exiting the interface. The CLI is fully interactive with all methods in the 
surf library and allows users to edit the fields in the configuration file and connect to 
\texttt{wpa\_supplicant}. 

\section*{Details of Completed Parts}
The only part of the project that reached total completion was the 
\texttt{Surf} library.
\texttt{Surf} interfaced with \texttt{wpa\_supplicant}'s own API, \texttt{wpa\_ctrl}
in order to edit the configuration file for \texttt{wpa\_supplicant} without the need
for root privileges.
Through \texttt{wpa\_ctrl}, \texttt{Surf} was able to support automatic configuration
for basic networks that rely on PSK's (pre-shared keys), or WPA-PSK networks,
and networks that rely on usernames and passwords, or WPA-EAP networks.
So in essence, the minimum amount of information required to connect to a specific 
network will typically be only the SSID of the network and its PSK, or the SSID
of a the network and a username and accompanying password.
In addition to automatic configuration, \texttt{Surf} supports editing network
configurations for advanced users, forgetting networks from known networks,
disabling and enabling networks without forgetting, and listing visible networks
in range of the connected wireless device and the known networks to \texttt{wpa\_supplicant}.
These features will allow a user interface implementer to worry only about presenting
information and getting information from the user instead of also dealing with
how to interface effectively with \texttt{wpa\_supplicant}. 
CLI has reached full completion on essential implementation of required parts. Users can 
fully interact with all methods in the surf API. Common commands users can enter include:
\begin{enumerate}
\item Connect to existing connections in the configuration file
\item Disconnect from an already existing connection
\item Add a new entry to the configuration file
\item Allow for both auto configuration and auto configuration of eap networks in the configuration file
\item Remove an already existing network in the configuration file
\item Edit an already existing network in the configuration file
\item List all configured networks in the configuration file
\item List all available networks
\item Connect to \texttt{wpa\_supplicant}
\item Check the current connection
\item Clear the terminal screen and exit program
\end{enumerate}
Commands for the CLI involve common words such as 'connect', 'disconnect', 'forget', 'edit', 
'current', 'list', 'configure', 'add', 'clear', and 'exit'. Help methods with the CLI are 
detailed and easy for users to understand and include the correct syntax for each command. 
CLI has required implementation for users to edit and interact with existing networks in the 
configuration file, connect and disconnect from already existing networks, list networks, in 
addition to descriptive help methods given with each command. 

\section*{Details of Incomplete Parts}
CLI does not grant users the ability to enter commands using hyphens for shortened commands. 
Additionally, multiple commands cannot be processed simultaneously. However, implementation for 
standard \texttt{getopt\_long} would require minimal implementation. Standard string comparsions 
are used for some commands where hyphens are detected for user input. Additional implementation 
would require \texttt{getopt\_long} to detect all commands with each function call in the command 
table. Additional implementation for listing the fields associated with an existing network in the 
configuration file, in addition to detecting whether or not a network connection is an eap network 
would be required for full completion of the CLI. Additional methods might be preferred by the 
developer to be considered complete. 
\subsection*{CLI}

%% TUI details here 
\subsection*{TUI}

%% GUI details here
\subsection*{GUI}
Since \texttt{Surf} was completed many weeks before the end of the senior project cycle,
the scope of the NetBSD Wifi Browser Project was expanded to include the development of 
GUI implementation for \texttt{Surf}.
In order to stay in compliance with NetBSD-base, the GUI was developed using the bare bones
\texttt{XLib} library, which is the basic graphics library included with \texttt{XOrg} 
(the default display environment on NetBSD).
This meant that development of the GUI would also include developing a library of graphics
elements to simplify the development of a GUI interface.
As such, combined with only 5 weeks of time left in the project, the GUI was only able
to reach a state of listing the known and available networks.
Progress was being made towards actually connecting to one of the available networks,
but was ultimately not achieved due to a lack of time and a complexity of the task
being performed.

\end{document}
