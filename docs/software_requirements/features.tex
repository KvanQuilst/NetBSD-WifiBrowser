\section{Features}
% This section contains a list of requirement statements.  This should be of the
% form "the system shall..."  and worded in a way that satisfaction of the
% requirement can be verified.  For example, the software shall authenticate
% users with at least two identification factors.  

\begin{enumerate}
  \item The API shall interface with \texttt{wpa\_supplicant} and IOCTL
  \item The API shall hash passkeys and passwords provided to it
  \item The API shall provide a manual process for connecting to a wifi network
  \item The API shall provide a semi-automatic process for connection to a wifi network
  \item The API shall provide a way to remove known/configured networks
  \item The CLI/TUI shall provide an interface to the API for the user
\end{enumerate}

%%%%%%%%%%%%%%%%%%%%%%%%%%%%%%%%%

% [For Each Feature]
\subsection{The API shall interface with \texttt{wpa\_supplicant} and IOCTL}
\subsubsection{Description}
% Describe the feature and how it fits into the overall product.  
In NetBSD, \texttt{wpa\_supplicant} and \texttt{IOCTL} are utilized to determine available wifi networks and
the specific details of a particular wifi network. The information from these are used to
construct a \texttt{wpa\_supplicant} configuration file, with which the \texttt{wpa\_supplicant} daemon is
then run, connecting the wifi interface to the wifi network. The API would streamline this
process, grabbing all of the necessary information from \texttt{wpa\_supplicant} and IOCTL
and putting it into a configuration file upon a network selection, requiring minimal
intervention from the user.
\subsubsection{Priority}
% Describe the relative importance of this feature. 
The feature is of high priority as it is the core of the purpose of the existence of this 
product.

\subsubsection{Stimulus and Response}
% What event will trigger the feature and how should the system respond.  This
% is probably an excerpt of a use case. 
This feature will be utilized every time a user needs to connect to a new network.
Upon it's use, the API will present the available (visible) networks and a network
will then need to be selected from that list. The API will then handle the configuration
file and restarting \texttt{wpa\_supplicant} for the connection to take affect.

\subsubsection{Functional Requirements}
% Formally state the functional requirement.
% The low-level format command shall require authorization by two
% system-administrators before beginning the low-level format operation.

\begin{itemize}
  \item The API shall require a call to it to gather available network information
  \item The API shall provide a list of available networks to connect to
  \item The API shall handle \texttt{wpa\_supplicant} configuration
  \item The API shall restart the \texttt{wpa\_supplicant} service (to read in new/appended 
  \item The API shall not require root privileges from the user
    config file)
\end{itemize}

%%%%%%%%%%%%%%%%%%%%%%%%%%%%%%%%%%%%%%%%%%%%%%

\subsection{The API shall hash passkeys and passwords}
\subsubsection{Description}
% Describe the feature and how it fits into the overall product.  
When using \texttt{wpa\_supplicant}, the default method of providing a passkey (to connect
to a network) or a password (for connecting to a network with a specific user account)
is be storing them in plain text in the configuration file. However, \texttt{wpa\_supplicant}
supports hashed passkeys and passwords in the configuration file, providing a
small additional layer to security. For passkeys, the \texttt{wpa\_supplicant} package provides
a hasher program. For passwords, piping a password through \texttt{iconv} and \texttt{openssl} provides
a hashed password.

\subsubsection{Priority}
% Describe the relative importance of this feature. 
This feature is important as it further protects a user's personal data and prevents
potentially malicious actors from having an easier entrance into a network.

\subsubsection{Stimulus and Response}
% What event will trigger the feature and how should the system respond.  This
% is probably an excerpt of a use case. 
This feature will be triggered any time the user inputs a passkey or a password.
Upon receiving a passkey, the API will run the passkey through the provided 
\texttt{wpa\_supplicant} hashing program before storing it in the configuration file. For
passwords, \texttt{wpa\_supplicant} will pipe the provided password through \texttt{iconv} and \texttt{openssl}
before storing the password in the configuration file.

\subsubsection{Functional Requirements}
% Formally state the functional requirement.
% The low-level format command shall require authorization by two
% system-administrators before beginning the low-level format operation.
\begin{itemize}
  \item The API shall hash passkeys and passwords before storage in the configuration file
\end{itemize}

%%%%%%%%%%%%%%%%%%%%%%%%%%%%%%%%%%%%%%%%%%

\subsection{The API shall provide a manual process for connecting to a network}
\subsubsection{Description}
% Describe the feature and how it fits into the overall product.  
In using this product, there may be a time that a user will want to manually configure
a network connection. Instead of writing a separate \texttt{wpa\_supplicant} configuration file,
the user will be able to provide the network information to the API and the API will handle
the actual interfacing with \texttt{wpa\_supplicant} and append it to the API's configuration file.

\subsubsection{Priority}
% Describe the relative importance of this feature. 
This feature is not of the highest priority as it does not actively work towards streamlining
the interaction with \texttt{wpa\_supplicant} and IOCTL in a significant way, but should be easily 
implementable with the existing infrastructure since the user will be providing the same
information that might be provided by \texttt{wpa\_supplicant} and \texttt{IOCTL}, and thus can be processed
similarly.

\subsubsection{Stimulus and Response}
% What event will trigger the feature and how should the system respond.  This
% is probably an excerpt of a use case. 
This feature will occur when the user invokes the call specifying there will be
manual input from the user. The user will provide at minimum the necessary information,
as well as any extra information for connecting to a network. If there is cause for more
information to be provided in the configuration, the API will grab the remaining necessary
information through a query with \texttt{wpa\_supplicant} or \texttt{IOCTL}.

\subsubsection{Functional Requirements}
% Formally state the functional requirement.
% The low-level format command shall require authorization by two
% system-administrators before beginning the low-level format operation.

\begin{itemize}
  \item The API will provide a library call for manual configuration
  \item The API will fill in any necessary information not provided by the user
    (the minimum provision being an SSID)
  \item The API will add the configuration information to its internal \texttt{wpa\_supplicant}
    configuration file
\end{itemize}

%%%%%%%%%%%%%%%%%%%%%%%%%%%%%%

\subsection{The API will provide a semi-automatic process for connecting to a wifi network}
\subsubsection{Description}
% Describe the feature and how it fits into the overall product.  
In NetBSD, much of the time spent connecting to a wifi network is spent identifying
the necessary information for the \texttt{wpa\_supplicant} configuration file. The API
can circumvent this by gathering the information automatically for the user
and automatically updating the internal \texttt{wpa\_supplicant} configuration file.

\subsubsection{Priority}
% Describe the relative importance of this feature. 
The priority of this high as this is the main feature that provides the streamlined
experience of connecting to a wifi network in NetBSD.

\subsubsection{Stimulus and Response}
% What event will trigger the feature and how should the system respond.  This
% is probably an excerpt of a use case. 
This feature will occur when the call for semi-automatic configuration is made
on the API. Upon invocation, the API will gather the SSIDs of the available 
(visible) networks and return them. The API will then require a network to be
selected from the list and the passkey or password associated with the connection
before gathering all the necessary information of the network from \texttt{wpa\_supplicant}
and \texttt{IOCTL}, appending it into the API's internal \texttt{wpa\_supplicant} configuration file.

\subsubsection{Functional Requirements}
% Formally state the functional requirement.
% The low-level format command shall require authorization by two
% system-administrators before beginning the low-level format operation.
\begin{itemize}
  \item The API shall return a list of available (visible) networks when requested
  \item The API shall handle the configuration of a \texttt{wpa\_supplicant} configuration file
\end{itemize}

%%%%%%%%%%%%%%%%%%%%%%%%%%%%%%

\subsection{The API shall provide a way to remove known/configured networks}
\subsubsection{Description}
% Describe the feature and how it fits into the overall product.  
There are many situations in which it makes sense to remove a network from the 
list of known/configured networks: don't need it anymore, contains potentially
sensitive information, or looking for minimal ways to save space. Whatever the
reason is, if a configured network can be added to the configuration file, then
a configured network can be removed.

\subsubsection{Priority}
% Describe the relative importance of this feature. 
This is of moderate priority, but is as simple as removing the lines that the
network occupies in the configuration profile.

\subsubsection{Stimulus and Response}
% What event will trigger the feature and how should the system respond.  This
% is probably an excerpt of a use case. 
Deletion of a network will be triggered when the user makes the call on the
API. At that point, the API will locate the network configurations position
in the configuration file and remove the lines that it occupies. When it is
finished, it will pass the modified configuration file to \texttt{wpa\_supplicant}.

\subsubsection{Functional Requirements}
% Formally state the functional requirement.
% The low-level format command shall require authorization by two
% system-administrators before beginning the low-level format operation.
\begin{itemize}
    \item The API will have the ability to remove a network configuration from the
      configuration file.
\end{itemize}

%%%%%%%%%%%%%%%%%%%%%%%%%%%%%%

\subsection{The CLI/TUI shall provide an interface to the API for the user}
\subsubsection{Description}
% Describe the feature and how it fits into the overall product.  
The API by itself is only useful if using as a library in source code of the
same language. Thus, a CLI/TUI will be provided as a way for a user to actually
interface with the API and its services. Through the interface, the user will be
able to take advantage of the API's features.

\subsubsection{Priority}
% Describe the relative importance of this feature. 
The priority of at least one of these interfaces is high as this will serve as the 
face of the product. Not many people will want to build their own interface on our
API library, so providing one that supplies all the necessary functionality is a must.

\subsubsection{Stimulus and Response}
% What event will trigger the feature and how should the system respond.  This
% is probably an excerpt of a use case. 
The CLI/TUI will be used any time a user needs to manage their network connections,
new or current. The interface will provided the tools necessary for interacting with
those networks and manage (most of) the heavy lifting for the user.

\subsubsection{Functional Requirements}
% Formally state the functional requirement.
% The low-level format command shall require authorization by two
% system-administrators before beginning the low-level format operation.
\begin{itemize}
    \item The CLI/TUI will provide the user with the ability to directly interface
      with the API
    \item The CLI/TUI will provide formatted presentation of the user's options
    \item The CLI/TUI will present formatted output from the API
\end{itemize}
