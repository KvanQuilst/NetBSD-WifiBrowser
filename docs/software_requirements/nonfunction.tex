\section{Other Nonfunctional Requirements}
% Nonfunctional requirements are shall-statements about how the software
% performs or is written.  It is not a statement about what the software does.
% For example, the function requirements for a Fibonacci function is this: the
% function shall return the nth Fibonacci number when provided n as input.  A
% nonfunctional requirement would be these: the function shall take time O(n)
% and all lines of the software shall be reachable by some test-case.
\begin{description}
  \item[Location of Configuration File] The \texttt{wpa\_supplicant} configuration file
    will be stored locally per user. This makes it so that root access is not
    required to modify the network configurations. With this, a central file
    will exist in conjunction containing "generic" versions of configured
    networks on the machine so that other users can utilize those configured
    networks, potentially with their own information (like with a WPA-EAP network).
\end{description}

\subsection{Performance Requirements}
% Your project will have performance requirements.  How long is the user willing
% to wait for the different features to execute?   For example, the software
% shall authenticate authorized users withing 250ms.   
\begin{itemize}
  \item In the API, automatic configuration should not take more than 1 second
    past the time it takes for the user to choose and network and enter the
    passkey or a username and password.
  \item After changing a configuration file, the configuration should not take
    more than 5-10 seconds to take effect in the \texttt{wpa\_supplicant} process.
\end{itemize}

\subsection{Security Requirements}
% Do not overlook this section.  Consider the three primary topics of computer
% security: Confidentiality, Integrity, and Availability.   Your project will
% have security requirements. 
\begin{description}
  \item[Hashing of Passkeys and Passwords] Passkeys will be hashed with the
    passkey hasher provided in the \texttt{wpa\_supplicant} package, and passwords will
    be hashed via a pipe through \texttt{iconv} to change the encoding to utf16 and piped
    to \texttt{openssl} to encrypt in an md4 hash. The purpose of this is to prevent storing
    passkeys and password in plain text, and offer an extra (albeit extremely thin)
    layer of security.
\end{description}

\subsection{Software Quality Attributes}
% Quality requirements can be the most difficult to write.  The desired quality
% must be written in a meaningful and testable way.  For example, All functions
% within the software shall have a cyclomatic number less than 10.  
\begin{itemize}
  \item The API will accurately determine and place configuration information
    in the \texttt{wpa\_supplicant} configuration file for a given network.
  \item The CLI/TUI will encapsulate all of the functionality that the API
    offers and way to easily interact with that functionality.
\end{itemize}
