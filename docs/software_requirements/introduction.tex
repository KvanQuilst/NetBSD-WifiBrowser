\section{Introduction}
% This section is used for multiple kinds of documents.  It describes the
% purpose of this document in the context of the rest of the product's
% documentation library.  

\subsection{Purpose}
% What is the purpose of this document as opposed to the vision and scope,
% software design document (492), testing plan (493), et al.

The purpose of this document is to introduce the core features of our API. Moreover, we will demonstrate the efficiency 
of having a streamlined process for connecting to Wi-Fi through NetBSD, as well as, functional and non-functional features 
of our API, security and performance aspects, as well as product description and interfaces. Each feature of our project is 
explicitly described in detail. While our vision and scope document appends the schedule and requirements for our project, this 
document describes what we want our API to achieve, and any limitations that might exist. 

\subsection{Document Conventions}
% Describe the conventions used in this document.  For example, throughout this
% document, the use of the plural shall imply the singular unless otherwise
% stated.  Now you can avoid parenthetical plurals like student(s).   

I don't believe any conventions are needed for this document. 

\subsection{Intended Audience and Reading Suggestions}
% Formally state the intended audience for this document.  For the SRS, the
% audience is usually developers, quality control, and documentation.  You are
% free to describe whatever readers you feel are appropriate, but you should not 
% describe the reader in terms of the class.  That is, do not refer to a teacher
% or other students.  

The intended audience includes developers, customers/stakeholders, and quality control.

\subsection{Project Scope}
% Put this developed project in the context of the overall product.  This is a
% brief summary of the vision and scope.  Just enough for the reader to
% understand the context for the requirements in later sections.

Vision and scope adheres to the requirements of our API. We plan to implement each core feature over the course of five milestones. 
Deliverables for our project include the API, command line interface and terminal user interface. Each milestone envelops a 
specific goal. Milestone one we expect to complete within’ the first five weeks of development, in which a skeleton of our project 
will be implemented. Milestone two and three will consist of the correct implementation of our product, including the CLI and TUI features. 
Milestone four and five will consist of security testing, in addition to meeting with stakeholders and customers as needed. We expect each 
milestone to be equally distributed in the amount of time it takes to complete. Individual responsibilities are equally distributed among 
team members. Project scope is only defined by the time it takes for each milestone to be completed. Resources are minimal, API development 
will be handled directly on NetBSD operating system. 

\subsection{References}
% List other documents that you refer to in the rest of this document.  Include
% unreferenced, but important documents for the project.  

The netbsd project. The NetBSD Project. (n.d.). Retrieved November 22, 2021, from https://www.netbsd.org/. 

