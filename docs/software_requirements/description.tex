\section{Overall Description}
% Describe the product's context in the larger business or industry setting.  Do
% not include specific features.  Give the reader an understand of how those
% features fit into the larger setting. 
%

\subsection{Product Perspective}
% How does the product fit in the business larger processes.  How is the user
% intended to fit the software into their business activities?   Consider
% including a figure that illustrates these relationships. 

This product fits into the business process by providing a more efficient means of connecting 
to Wi-Fi. For people who use NetBSD, we can reduce the time that is taken to enter commands manually 
through \texttt{wpa\_client}. Currently, commands are entered through the terminal, and many different 
commands are needed. Such commands involve searching for network interfaces, accessing the network 
interface, scanning the interface and then connecting to a specific Wi-Fi connection through that 
interface. We wish to reduce the time needed to do this by creating an API that can streamline this 
process for us. I don’t believe a figure is needed for this relationship, because our API will simply 
reduce the time needed for a process that already exists. 

\subsection{Product Features}
% This is a list of high-level description of the functional behavior of the
% product.  This should give the reader a better understanding of how the formal
% requirements fit together. 

The following features of our Wi-Fi API involve the following: 

\begin{enumerate}
\item An interface that interacts with \texttt{wpa\_supplicant} and IOCTL 
\item The hashing of passkeys and passwords 
\item A manual process for connecting to Wi-Fi networks 
\item A semi-automatic process for connecting to Wi-Fi networks 
\item A process for removing known/configured networks 
\item A client and text interface for the user. 
\end{enumerate} 

Each feature of our product implements a specific component of our API and is fairly descriptive on their own. 
The interface that interacts with \texttt{wpa\_supplicate} and IOCTL will involve a process that is similar to 
windows. Where a user can interact with the peripherals of their device to connect to a Wi-Fi network, without 
inputting specific commands into \texttt{wpa\_client}. The hashing of passkeys and passwords is typically done 
through a configuration file on the system. However, our API will implement this feature by running the passkey 
and password and storing them in the configuration file through our API. For establishing manual network connections, 
we will create a process where the user can manually configure a network where our API will then append the new network 
connection to the configuration file of the system. \\
The semi-automatic feature is fairly straight forward in that our API simply processes the information already stored in the 
configuration file. For removing known / configured networks, our API simply accesses the information in the configuration file 
and simply removes the data for the specific network. This involves developing a process that is the reverse of establishing 
a manual connection through Wi-Fi, since our API will be removing data instead of adding it to the file. We will also need 
to establish as least a client terminal interface and text editor user interface for our API. While we hope to eventually 
establish a GUI component of our interface, the client terminal interface and text editor interface will provide the user with a 
method of interacting with our API.

\subsection{User Classes and Characteristics}
% Describe the different rolls or classes of users.  For each user class,
% describe the user class's principal characteristics.  For example, unix
% systems have at least two classes of users: system administrators and
% operators.  

\subsection{Operating Environment}
% What is the expected environment?  For example, the product could be a desktop
% application with users who work in a formal office environment.  Contrast this
% with a mobile application for mountain biking that keeps track of GPS locations.

\subsection{Design and Implementation Constraints}
% List any constraints that are part of the project.  For example, health
% services applications must implement HIPAA regulations.  

\subsection{Assumptions and Dependencies}
% List assumptions and dependencies that are not formal constraints.  Items in
% this list will, if changed, will cause a change in the formal requirements in
% the next section.
%
