\section{Overall Description}
% Describe the product's context in the larger business or industry setting.  Do
% not include specific features.  Give the reader an understand of how those
% features fit into the larger setting. 
%
The NetBSD Wifi Browser is to fill a void that exists within the NetBSD operating
system: the need for a streamlined method to connect to wifi. At the current
moment, wifi connection requires a lot of manual effort and configuration to
achieve a connection. The NetBSD Wifi Browser will simplify this process
by automating some of the manual configuration bits, significantly reducing the
effort and time required of the user.

\subsection{Product Perspective}
% How does the product fit in the business larger processes.  How is the user
% intended to fit the software into their business activities?   Consider
% including a figure that illustrates these relationships. 
This product will be particularly beneficial when a user needs to connect to
a new network, a common occurrence with fresh installs of NetBSD and when the
computer with NetBSD has moved to a new location.

\subsection{Product Features}
% This is a list of high-level description of the functional behavior of the
% product.  This should give the reader a better understanding of how the formal
% requirements fit together. 
The NetBSD Wifi Browser will be comprised of two components: an API and a
user interface. The API will handle the interactions with \texttt{wpa\_supplicant}
and be the source of automation of network configuration, with the only required
input from the user being the network (chosen from a list typically) and the
passkey or username and password for that network. The UI then will utilize the
API to allow a user to take advantage of the features the API will contain.

\subsection{User Classes and Characteristics}
% Describe the different rolls or classes of users.  For each user class,
% describe the user class's principal characteristics.  For example, unix
% systems have at least two classes of users: system administrators and
% operators.  
\begin{description}
  \item[General User] The general user will have a desire to easily connect
    to wifi networks and remove known networks.
  \item[Developer] The developer may use the API in their own development projects
    or in an alternative user interface.
\end{description}

\subsection{Operating Environment}
% What is the expected environment?  For example, the product could be a desktop
% application with users who work in a formal office environment.  Contrast this
% with a mobile application for mountain biking that keeps track of GPS locations.
The expected operating environment is the NetBSD operating system. We are constructing an application program interface to 
interact with this specific type of operating system. This is not an API that can run on mobile devices, and is a standard 
desktop API used for making the process of connecting to existing Wi-Fi connections much simpler. 

\subsection{Design and Implementation Constraints}
% List any constraints that are part of the project.  For example, health
% services applications must implement HIPAA regulations.  
The intended environment for the NetBSD Wifi Browser is on a NetBSD desktop,
initially only within a terminal as the only provided user interfaces will be
a CLI or TUI.

\subsection{Assumptions and Dependencies}
% List assumptions and dependencies that are not formal constraints.  Items in
% this list will, if changed, will cause a change in the formal requirements in
% the next section.
%
The NetBSD Wifi Browser API will heavily depend on \texttt{wpa\_supplicant} and its
provided interfaces. The API will be designed and implemented under the assumption
that \texttt{wpa\_supplicant} and its interfaces will not change its policies.
