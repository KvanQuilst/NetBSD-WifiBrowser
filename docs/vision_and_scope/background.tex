\section{Business Requirements}
\subsection{Background}
% What is the business context for the project?
% What does the reader need to know about the customer's industry?

In NetBSD, in order to connect to Wi-Fi the user must manually connect using WPA Supplicant and IOCTL to determine the Wi-Fi information. 
We hope to create an API that will streamline the process of connecting to Wi-Fi. In windows, users connect to Wi-Fi using an API, in which 
the user does not need to type specific commands into the command line interface. The business context involves creating an API that will make 
the process of connecting to Wi-Fi much easier for the user. 

\subsection{Business Opportunity}
% What is the customer's need?
% How does this need fit in the industrial context?

The customer’s need is a streamlined method to connect devices to Wi-Fi, this would be helpful because NetBSD has no available options to do this. 
This could potentially save time for the average user upon loading the system. An API that would make the process much simpler for the user. 

\subsection{Business Objectives and Success Criteria}
% Why does the customer need this product?
% How will the customer know that the product is a success?
%   You must be very specific here.  What experiment can we do to verify success?

The customer needs this product to help minimize the time and effort spent connecting to Wi-Fi. The customer will know the product is a success if 
the user is able to effortlessly connect to Wi-Fi in less time. Creating an API to streamline the process would take less effort for the typical user. 
We can determine if our product is a success by thoroughly testing it before deployment. If we can effortlessly connect to Wi-Fi, using our API, we will 
consider our project a success in this context. 

\subsection{Customer and Market Needs}
% Connect the objectives and success criteria back to the customer's business.
% That is, why does the industry require the customer to have this solution? 

The industry requires this solution because a more efficient method of connecting to Wi-Fi is needed in the NetBSD operating system. Customers and 
stakeholders alike would prefer a more efficient solution for connecting to Wi-Fi through the NetBSD operating system. NetBSD would have an additional 
API that would make it’s process for setting network connections much simpler. This is a preference to most customers as it is often time consuming to 
connect through Wi-Fi every time through the command line interface. 

\subsection{Business Risks}
% This is a hazard assessment.  What could go wrong?  How bad would it be if it
% did?  How likely is it?  What steps can we take to protect against the hazard? 

We can determine our biggest failure if our API is not efficient in connecting to Wi-Fi, or if the API does not work altogether. However, business risks 
and hazard is very low. We will thoroughly test our product before deployment to ensure it works before customers use the new system. If our API does not work, 
customers will still be able to connect to Wi-Fi using the wpa client. This puts our risk very low as our API will only work to improve a system where no current 
solution exists. We can work to protect against this by thoroughly testing our product before deployment. 

