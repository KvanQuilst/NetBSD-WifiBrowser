\section{Scope and Limitations}
% This section identifies what part of the envisioned solution is going to be
% developed in this current project or version, what will be delayed until
% latter projects, and what is out of scope for the whole product.  
%
Within the scope of the next 8 months in which we have to work, we will accomplish three major tasks: the creation of an API which interfaces with
wpa\_supplicant and IOCTL, a command line interface built on top of the API, and a terminal user interface also built on the API, both of which will
allow a user to easily interact with wpa\_supplicant to connect to a wifi network. If time and development progress is permitting, a GUI in the 
style of one of the popular desktop environments of NetBSD will also be created similarly to the CLI and the TUI. An aspect that will have to wait
for later projects and probably other teams is the development of GUI interfaces for each of the popular desktop environments. Out of scope for
the wifi browser project is the porting of the applications and API to other UNIX or BSD based operating systems.

\subsection{Scope of Initial Release}

Our initial release of the product will certainly feature the API and the command line interface. This will meet the bare minimum of the product
requirements and provide a foundational perspective of the future of the product, as well as inform the changes that may be needed from either our
API or CLI.

\subsection{Scope of Subsequent Releases}

The release immediate proceeding from the initial release will be the creation of a terminal-base user interface. Subsequent releases will most
likely focus on the implementation of proper GUI's built on the API, and, in the event that wpa\_supplicant or IOCTL change significantly, updating
our API to continue proper interfacing with both programs.

\subsection{Limitations and Exclusions}

Our major limitations for this project is the team having minimal experience with the NetBSD operating system and the amount of time we will
have to work towards our goal. On top of this limitation, we will need to allot time to understand how to use wpa\_supplicant and IOCTL to 
connect to wifi networks of all standards and protocols so that we properly design an API to do that for us.
