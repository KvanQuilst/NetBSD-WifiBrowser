\documentclass[11pt]{article}
\setlength{\parindent}{0em}
\usepackage[margin=1in]{geometry}

\begin{document}

\begin{center}
  \textbf{\Large Milestone \#4}\\\large NetBSD Wifi Browser Project 2022\\
  Dylan Roy, Stephen Loudiana, Kevin McGrane
\end{center}


\section{General Progress Report}
Progress in the API was brought to a version 1.0 fully functional state during
the progress of Milestone 5. During this time, the API removed the use of file
I/O, making it possible to use the API without root permissions if operating on
a configuration file that is not accessible without root permission. The API is
currently in development for v1.01 which will feature cleaner output, especially
when listing scanned networks and configured networks, making it easier for user
interface implementers to utilize the data passed to it.\\

\textbf{Accomplishments}
\begin{itemize}
  \item Wifi Browser API v1.0 is finished and fully functional!
\end{itemize}

\textbf{Targets Not Met:}
\begin{itemize}
  \item
\end{itemize}


\section{Git Branch Report}
One additional branch was used to clean up the output provided by \texttt{wpa\_supplicant}
(branch \texttt{output}) to preserve the functionality of the API in v1.0 state.
Beyond this, most work was performed on \texttt{master}.

\section{Individual Progress}

\textbf{Kevin:}
\begin{itemize}
  \item
\end{itemize}

\textbf{Dylan:}
\begin{itemize}
  \item Finished API to v1.0
  \item Converted all configuration manipulation to \texttt{wpa\_supplicant}'s
    provided interface (\texttt{wpa\_ctrl})
  \item Removed the necessity for root permissions in using the API
  \item Reimplemented editing of network configurations to be more expandable
    and easier for developers to utilize
  \item Implemented auto WPA-EAP network configuration
  \item Started \texttt{wpa\_supplicant} output cleanup for easier use in user
    interface implementation.
\end{itemize}

\textbf{Stephen:}
\begin{itemize}
  \item
\end{itemize}

\end{document}




