\documentclass[11pt]{article}
\setlength{\parindent}{0em}
\usepackage[margin=1in]{geometry}

\begin{document}

\begin{center}
  \textbf{\Large Milestone \#2}\\\large NetBSD Wifi Browser Project 2022\\
  Dylan Roy, Stephen Loudiana, Kevin McGrane
\end{center}


\section{General Progress Report}
Progress has been significant for the duration of Milestone 2, but not necessarily in the ways outlined
in the development plan. Instead of all members focusing on the API to start out, we split the work into
the three main components our projects encompasses: Dylan on the API, Stephen on the CLI, and Kevin on the TUI.
The API's progress had many road blocks in its development due to the numerous undocumented quirks of interfacing
with \texttt{wpa\_supplicant}, though, as the roadblocks are overcome, progress on the API should speed up
as the implementation beyond the quirks is rather trivial. In the development plan document, the next milestone
might look a bit more ambitious, but the majority of the API implementation planned is closely related to each
other and \textit{should} have trivial implementation once the configuration file saving issue is solved.
Making up for the unexpected delays in the API, Stephen has put a significant amount of work into the CLI, saving 
us time that was allocated for later in the quarter originally, negating (hopefully) the API slow downs.

The progress on the TUI has been slow to get started due to the fact that it has mostly been filled with research 
and comparing with other TUIs to see the best plan of action when trying to implement it. It seems that Ncurses 
is going to be our best bet when trying to efficiently implement a TUI that works with the API, thus a large majority 
of the our time during milestone has been used on researching Ncurses and how to use it when implementing the TUI. 
There was also some confusion at the start of our implementation on how the TUI will interact and work with the API, 
but after a few group discussion those gray areas were cleared up and have really set us up for success moving into 
the future.

\textbf{Accomplishments}
\begin{itemize}
  \item API can communicate with a running instance of \texttt{wpa\_supplicant}
  \item Projects can use the API in its current state and use some functionality
\end{itemize}

\textbf{Targets Not Met:}
\begin{itemize}
  \item API has not independently facilitated \texttt{wpa\_supplicant} connecting to a wifi source of any kind
  \item API is not hashing passwords yet
  \item API can remove networks from configuration, but not externally yet
  \item API testing is not really a suite yet, but still being built out
\end{itemize}


\section{Git Branch Report}
All work has been performed on \texttt{master} against best practice. It might actually behoove us to
work on separate branches due to the experienced difficulties with people working on master at the same
time.


\section{Individual Progress}

\textbf{Stephen:} <time>
\begin{itemize}

\end{itemize}


\textbf{Kevin:} <time>
\begin{itemize}
  \item Research on best way to implement TUI by comparing with other examples and working TUI systems
  \item Research on Ncurses and how it will be implemented 
  \item Began some sample testing with Ncurses 
  \item Review of API .h file and how it is going to interact with the TUI
  \item Research and problem solving on already existing issues with Ncurses when implementing a TUI
\end{itemize}


\textbf{Dylan:} <time>
\begin{itemize}
  \item Set up remote connection to lab machine
  \item Built API scaffolding and function prototypes
  \item Extensive additional into interfacing with \texttt{wpa\_supplicant}
  \item Linked \texttt{wpa\_ctrl} - the \texttt{wpa\_supplicant} interface - to the API
  \item API can compile with \texttt{wpa\_ctrl} files and a program file
  \item Connected the API to \texttt{wpa\_supplicant}
  \item API can list available networks in proximity to the wifi interface
  \item API can list configured networks in the configuration file \texttt{wpa\_supplicant} is currently using
  \item API can temporarily create new network configurations and write to them, but they don't save when 
    \texttt{wpa\_supplicant} is asked to re-read the configuration file
  \item API can create a local configuration file, but \texttt{wpa\_supplicant} terminate when trying to use it mysteriously
  \item Building out a document of interfacing with \texttt{wpa\_supplicant}
\end{itemize}


\end{document}
